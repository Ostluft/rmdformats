% Options for packages loaded elsewhere
\PassOptionsToPackage{unicode}{hyperref}
\PassOptionsToPackage{hyphens}{url}
%
\documentclass[
]{article}
\usepackage{lmodern}
\usepackage{amssymb,amsmath}
\usepackage{ifxetex,ifluatex}
\ifnum 0\ifxetex 1\fi\ifluatex 1\fi=0 % if pdftex
  \usepackage[T1]{fontenc}
  \usepackage[utf8]{inputenc}
  \usepackage{textcomp} % provide euro and other symbols
\else % if luatex or xetex
  \usepackage{unicode-math}
  \defaultfontfeatures{Scale=MatchLowercase}
  \defaultfontfeatures[\rmfamily]{Ligatures=TeX,Scale=1}
\fi
% Use upquote if available, for straight quotes in verbatim environments
\IfFileExists{upquote.sty}{\usepackage{upquote}}{}
\IfFileExists{microtype.sty}{% use microtype if available
  \usepackage[]{microtype}
  \UseMicrotypeSet[protrusion]{basicmath} % disable protrusion for tt fonts
}{}
\makeatletter
\@ifundefined{KOMAClassName}{% if non-KOMA class
  \IfFileExists{parskip.sty}{%
    \usepackage{parskip}
  }{% else
    \setlength{\parindent}{0pt}
    \setlength{\parskip}{6pt plus 2pt minus 1pt}}
}{% if KOMA class
  \KOMAoptions{parskip=half}}
\makeatother
\usepackage{xcolor}
\IfFileExists{xurl.sty}{\usepackage{xurl}}{} % add URL line breaks if available
\IfFileExists{bookmark.sty}{\usepackage{bookmark}}{\usepackage{hyperref}}
\hypersetup{
  pdftitle={skeleton ostluft/readthedown with pandoc-crossref},
  pdfauthor={Your Name},
  hidelinks,
  pdfcreator={LaTeX via pandoc}}
\urlstyle{same} % disable monospaced font for URLs
\usepackage{color}
\usepackage{fancyvrb}
\newcommand{\VerbBar}{|}
\newcommand{\VERB}{\Verb[commandchars=\\\{\}]}
\DefineVerbatimEnvironment{Highlighting}{Verbatim}{commandchars=\\\{\}}
% Add ',fontsize=\small' for more characters per line
\usepackage{framed}
\definecolor{shadecolor}{RGB}{255,255,255}
\newenvironment{Shaded}{\begin{snugshade}}{\end{snugshade}}
\newcommand{\AlertTok}[1]{\textcolor[rgb]{0.75,0.01,0.01}{\textbf{\colorbox[rgb]{0.97,0.90,0.90}{#1}}}}
\newcommand{\AnnotationTok}[1]{\textcolor[rgb]{0.79,0.38,0.79}{#1}}
\newcommand{\AttributeTok}[1]{\textcolor[rgb]{0.00,0.34,0.68}{#1}}
\newcommand{\BaseNTok}[1]{\textcolor[rgb]{0.69,0.50,0.00}{#1}}
\newcommand{\BuiltInTok}[1]{\textcolor[rgb]{0.39,0.29,0.61}{\textbf{#1}}}
\newcommand{\CharTok}[1]{\textcolor[rgb]{0.57,0.30,0.62}{#1}}
\newcommand{\CommentTok}[1]{\textcolor[rgb]{0.54,0.53,0.53}{#1}}
\newcommand{\CommentVarTok}[1]{\textcolor[rgb]{0.00,0.58,1.00}{#1}}
\newcommand{\ConstantTok}[1]{\textcolor[rgb]{0.67,0.33,0.00}{#1}}
\newcommand{\ControlFlowTok}[1]{\textcolor[rgb]{0.12,0.11,0.11}{\textbf{#1}}}
\newcommand{\DataTypeTok}[1]{\textcolor[rgb]{0.00,0.34,0.68}{#1}}
\newcommand{\DecValTok}[1]{\textcolor[rgb]{0.69,0.50,0.00}{#1}}
\newcommand{\DocumentationTok}[1]{\textcolor[rgb]{0.38,0.47,0.50}{#1}}
\newcommand{\ErrorTok}[1]{\textcolor[rgb]{0.75,0.01,0.01}{\underline{#1}}}
\newcommand{\ExtensionTok}[1]{\textcolor[rgb]{0.00,0.58,1.00}{\textbf{#1}}}
\newcommand{\FloatTok}[1]{\textcolor[rgb]{0.69,0.50,0.00}{#1}}
\newcommand{\FunctionTok}[1]{\textcolor[rgb]{0.39,0.29,0.61}{#1}}
\newcommand{\ImportTok}[1]{\textcolor[rgb]{1.00,0.33,0.00}{#1}}
\newcommand{\InformationTok}[1]{\textcolor[rgb]{0.69,0.50,0.00}{#1}}
\newcommand{\KeywordTok}[1]{\textcolor[rgb]{0.12,0.11,0.11}{\textbf{#1}}}
\newcommand{\NormalTok}[1]{\textcolor[rgb]{0.12,0.11,0.11}{#1}}
\newcommand{\OperatorTok}[1]{\textcolor[rgb]{0.12,0.11,0.11}{#1}}
\newcommand{\OtherTok}[1]{\textcolor[rgb]{0.00,0.43,0.16}{#1}}
\newcommand{\PreprocessorTok}[1]{\textcolor[rgb]{0.00,0.43,0.16}{#1}}
\newcommand{\RegionMarkerTok}[1]{\textcolor[rgb]{0.00,0.34,0.68}{\colorbox[rgb]{0.88,0.91,0.97}{#1}}}
\newcommand{\SpecialCharTok}[1]{\textcolor[rgb]{0.24,0.68,0.91}{#1}}
\newcommand{\SpecialStringTok}[1]{\textcolor[rgb]{1.00,0.33,0.00}{#1}}
\newcommand{\StringTok}[1]{\textcolor[rgb]{0.75,0.01,0.01}{#1}}
\newcommand{\VariableTok}[1]{\textcolor[rgb]{0.00,0.34,0.68}{#1}}
\newcommand{\VerbatimStringTok}[1]{\textcolor[rgb]{0.75,0.01,0.01}{#1}}
\newcommand{\WarningTok}[1]{\textcolor[rgb]{0.75,0.01,0.01}{#1}}
\usepackage{longtable,booktabs}
% Correct order of tables after \paragraph or \subparagraph
\usepackage{etoolbox}
\makeatletter
\patchcmd\longtable{\par}{\if@noskipsec\mbox{}\fi\par}{}{}
\makeatother
% Allow footnotes in longtable head/foot
\IfFileExists{footnotehyper.sty}{\usepackage{footnotehyper}}{\usepackage{footnote}}
\makesavenoteenv{longtable}

% OSTLUFT: needed
\usepackage[margin=1in]{geometry}
\usepackage{float}

\usepackage{graphicx}
\makeatletter
\def\maxwidth{\ifdim\Gin@nat@width>\linewidth\linewidth\else\Gin@nat@width\fi}
\def\maxheight{\ifdim\Gin@nat@height>\textheight\textheight\else\Gin@nat@height\fi}
\makeatother
% Scale images if necessary, so that they will not overflow the page
% margins by default, and it is still possible to overwrite the defaults
% using explicit options in \includegraphics[width, height, ...]{}
\setkeys{Gin}{width=\maxwidth,height=\maxheight,keepaspectratio}
% Set default figure placement to htbp
\makeatletter
\def\fps@figure{htbp}
\makeatother
\usepackage[normalem]{ulem}
% Avoid problems with \sout in headers with hyperref
\pdfstringdefDisableCommands{\renewcommand{\sout}{}}
\setlength{\emergencystretch}{3em} % prevent overfull lines
\providecommand{\tightlist}{%
  \setlength{\itemsep}{0pt}\setlength{\parskip}{0pt}}
\setcounter{secnumdepth}{5}
\makeatletter
\@ifpackageloaded{subfig}{}{\usepackage{subfig}}
\@ifpackageloaded{caption}{}{\usepackage{caption}}
\captionsetup[subfloat]{margin=0.5em}
\AtBeginDocument{%
\renewcommand*\figurename{Abbildung}
\renewcommand*\tablename{Tabelle}
}
\AtBeginDocument{%
\renewcommand*\listfigurename{List of Figures}
\renewcommand*\listtablename{List of Tables}
}
\@ifpackageloaded{float}{}{\usepackage{float}}
\floatstyle{ruled}
\@ifundefined{c@chapter}{\newfloat{codelisting}{h}{lop}}{\newfloat{codelisting}{h}{lop}[chapter]}
\floatname{codelisting}{Code Block}
\newcommand*\listoflistings{\listof{codelisting}{List of Listings}}
\makeatother

\title{skeleton ostluft/readthedown with pandoc-crossref}
\author{Your Name}
\date{Created: 14.04.2020, Updated: 21.04.2020}

\begin{document}


% OSTLUFT: use exact position of figure
\makeatletter\renewcommand*{\fps@figure}{H}\makeatother

% OSTLUFT: insert the images
\begin{figure}[t]
\subfloat{\includegraphics[width=132pt]{ostluft-logo.png}}
\hfill
\subfloat{\includegraphics[width=338pt]{ostluft-wappen.png}}
\end{figure}

\maketitle

\hypertarget{features}{%
\section{Features}\label{features}}

\hypertarget{cross-referencing}{%
\subsection{Cross referencing}\label{cross-referencing}}

The skeleton is using
\href{https://github.com/lierdakil/pandoc-crossref}{lierdakil/pandoc-crossref}.
This requires the installation of pandoc-crossref with a compatible
version of pandoc. Additional a knitr hook to add the necessaries labels
is needed.

In Addition the template supports
\href{https://github.com/rstudio/bookdown}{bookdown}. Use
\texttt{use\_bookdown:\ TRUE} in the yaml header to use the
\texttt{bookdown::html\_document2()} renderer.

Both solutions have their merits and disadvantages. pandoc-crossref is
more configurable, but also needs more setup. bookdown one the other
hand has
\href{https://bookdown.org/yihui/bookdown/motivation.html}{additionals
features}.

\hypertarget{lightbox-support}{%
\subsection{lightbox support}\label{lightbox-support}}

Using the \href{http://dimsemenov.com/plugins/magnific-popup/}{Magnific
popup} lightbox plugin for plots and images. To enable the plugin set
\texttt{lightbox:\ TRUE} and for gallery support \texttt{gallery:\ TRUE}
in the yaml header. Fore more infos see Kapitel~\ref{sec:lb-support}.

\hypertarget{minimal-pdf-support}{%
\subsection{minimal pdf support}\label{minimal-pdf-support}}

The file \texttt{ostluft\_header.tex} includes the minimal changes to
create a pdf output. Search for \texttt{\%\ OSTLUFT} to inspect the
changes.

\hypertarget{sec:crossref}{%
\section{Cross Referencing}\label{sec:crossref}}

\hypertarget{sec:test}{%
\subsection{Test Reference}\label{sec:test}}

\begin{itemize}
\tightlist
\item
  Reference plot Abbildung~\ref{fig:plt-pressure}
\item
  Reference plots Abbildung~\ref{fig:plt-multi}
\item
  Reference image Abbildung~\ref{fig:ostluft-logo}
\item
  Reference image Abbildung~\ref{fig:knitr-include-graphics}
\item
  Reference plotted image Abbildung~\ref{fig:img-vulcan}
\item
  Reference chapter Kapitel~\ref{sec:test}
\item
  Reference equation Gleichung~\ref{eq:quadratic_formula}
\item
  Reference to a \texttt{kable} table Tabelle~\ref{tbl:kable}
\item
  Reference to a Code Block Code Block~\ref{lst:syntax_highlighting}
\item
  Reference to a Code Block Code Block~\ref{lst:echo}
\end{itemize}

\hypertarget{plots}{%
\subsection{plots}\label{plots}}

\hypertarget{simple-single-plot}{%
\subsubsection{simple single plot}\label{simple-single-plot}}

\begin{figure}
\hypertarget{fig:plt-pressure}{%
\centering
\includegraphics{readthedown_files/figure-latex/plt-pressure-1.pdf}
\caption{Vapor Pressure of Mercury as a Function of
Temperature}\label{fig:plt-pressure}
}
\end{figure}

\hypertarget{multiple-plots}{%
\subsubsection{multiple plots}\label{multiple-plots}}

\includegraphics{readthedown_files/figure-latex/plt-multi-1.pdf}

\begin{figure}
\hypertarget{fig:plt-multi}{%
\centering
\includegraphics{readthedown_files/figure-latex/plt-multi-2.pdf}
\caption{Vapor Pressure of Mercury as a Function of
Temperature}\label{fig:plt-multi}
}
\end{figure}

\hypertarget{image-function-works-too}{%
\subsubsection{image function works
too}\label{image-function-works-too}}

\begin{figure}
\hypertarget{fig:img-vulcan}{%
\centering
\includegraphics{readthedown_files/figure-latex/img-vulcan-1.pdf}
\caption{Vulkan mit heat.colors(200)}\label{fig:img-vulcan}
}
\end{figure}

\protect\hyperlink{sec:test}{Back to Test Links}

\hypertarget{external-image}{%
\subsection{external image}\label{external-image}}

\hypertarget{markdown-syntax}{%
\subsubsection{Markdown Syntax}\label{markdown-syntax}}

\begin{figure}
\hypertarget{fig:ostluft-logo}{%
\centering
\includegraphics[width=0.54167in,height=\textheight]{ostluft-logo.png}
\caption{Ostluft Logo}\label{fig:ostluft-logo}
}
\end{figure}

\hypertarget{knitrinclude_graphics}{%
\subsubsection{knitr::include\_graphics}\label{knitrinclude_graphics}}

\begin{figure}
\hypertarget{fig:knitr-include-graphics}{%
\centering
\includegraphics[width=2.90625in,height=\textheight]{ostluft-wappen.png}
\caption{Ostluft Logo with
\texttt{knitr::include\_graphics}}\label{fig:knitr-include-graphics}
}
\end{figure}

\protect\hyperlink{sec:test}{Back to Test Links}

\hypertarget{no-cross-references}{%
\subsection{NO cross references}\label{no-cross-references}}

\hypertarget{dont-include-chunks-without-a-label}{%
\subsubsection{dont include chunks without a
label}\label{dont-include-chunks-without-a-label}}

\begin{figure}
\centering
\includegraphics{readthedown_files/figure-latex/unnamed-chunk-1-1.pdf}
\caption{Average Heights and Weights for American Women}
\end{figure}

\hypertarget{dont-include-chunks-with-options-crossreffalse}{%
\subsubsection{dont include chunks with options
crossref=FALSE}\label{dont-include-chunks-with-options-crossreffalse}}

\begin{figure}
\centering
\includegraphics{readthedown_files/figure-latex/plt-sleep-prolongation-1.pdf}
\caption{Sleep prolongation}
\end{figure}

\protect\hyperlink{sec:test}{Back to Test Links}

\hypertarget{markdown-format-syntax-examples}{%
\section{Markdown Format \& Syntax
Examples}\label{markdown-format-syntax-examples}}

\hypertarget{code-and-tables}{%
\subsection{Code and tables}\label{code-and-tables}}

\hypertarget{syntax-highlighting}{%
\subsubsection{Syntax highlighting}\label{syntax-highlighting}}

Here is a sample code chunk, just to show that syntax highlighting works
as expected.

\begin{codelisting}

\caption{Syntax highlighting}

\hypertarget{lst:syntax_highlighting}{%
\label{lst:syntax_highlighting}}%
\begin{Shaded}
\begin{Highlighting}[]
\KeywordTok{library}\NormalTok{(ggplot2)}
\KeywordTok{library}\NormalTok{(dplyr)}
\NormalTok{not\_null \textless{}{-}}\StringTok{ }\ControlFlowTok{function}\NormalTok{(v) \{}
    \ControlFlowTok{if}\NormalTok{ (}\OperatorTok{!}\KeywordTok{is.null}\NormalTok{(v)) }
        \KeywordTok{return}\NormalTok{(}\KeywordTok{paste}\NormalTok{(v, }\StringTok{"not null"}\NormalTok{))}
\NormalTok{\}}
\KeywordTok{data}\NormalTok{(iris)}
\NormalTok{tab \textless{}{-}}\StringTok{ }\NormalTok{iris }\OperatorTok{\%\textgreater{}\%}\StringTok{ }\KeywordTok{group\_by}\NormalTok{(Species) }\OperatorTok{\%\textgreater{}\%}\StringTok{ }\KeywordTok{summarise}\NormalTok{(}\DataTypeTok{Sepal.Length =} \KeywordTok{mean}\NormalTok{(Sepal.Length), }\DataTypeTok{Sepal.Width =} \KeywordTok{mean}\NormalTok{(Sepal.Width), }\DataTypeTok{Petal.Length =} \KeywordTok{mean}\NormalTok{(Petal.Length), }
    \DataTypeTok{Petal.Width =} \KeywordTok{mean}\NormalTok{(Petal.Length))}
\end{Highlighting}
\end{Shaded}

\end{codelisting}

\hypertarget{verbatim}{%
\subsubsection{Verbatim}\label{verbatim}}

Here is the structure of the \texttt{iris} dataset.

\begin{Shaded}
\begin{Highlighting}[]
\KeywordTok{str}\NormalTok{(iris)}
\end{Highlighting}
\end{Shaded}

\begin{codelisting}

\caption{\texttt{ECHO=TRUE}}

\hypertarget{lst:echo}{%
\label{lst:echo}}%
\begin{verbatim}
'data.frame':   150 obs. of  5 variables:
 $ Sepal.Length: num  5.1 4.9 4.7 4.6 5 5.4 4.6 5 4.4 4.9 ...
 $ Sepal.Width : num  3.5 3 3.2 3.1 3.6 3.9 3.4 3.4 2.9 3.1 ...
 $ Petal.Length: num  1.4 1.4 1.3 1.5 1.4 1.7 1.4 1.5 1.4 1.5 ...
 $ Petal.Width : num  0.2 0.2 0.2 0.2 0.2 0.4 0.3 0.2 0.2 0.1 ...
 $ Species     : Factor w/ 3 levels "setosa","versicolor",..: 1 1 1 1 1 1 1 1 1 1 ...
\end{verbatim}

\end{codelisting}

And blockquote :

\begin{quote}
Oh ! What a nice blockquote you have here. Much more wonderful than a
classical Lorem Ipsum, really. And we could also
\href{https://github.com/juba/rmdformats}{include links} or simply URLs
like this : \url{https://github.com/juba/rmdformats}
\end{quote}

\hypertarget{table}{%
\subsubsection{Table}\label{table}}

Here is a sample table output.

\hypertarget{tbl:kable}{}
\begin{longtable}[]{@{}lrrrr@{}}
\caption{\label{tbl:kable}\texttt{kable} cross reference
example}\tabularnewline
\toprule
Species & Sepal.Length & Sepal.Width & Petal.Length &
Petal.Width\tabularnewline
\midrule
\endfirsthead
\toprule
Species & Sepal.Length & Sepal.Width & Petal.Length &
Petal.Width\tabularnewline
\midrule
\endhead
setosa & 5.006 & 3.428 & 1.462 & 1.462\tabularnewline
versicolor & 5.936 & 2.770 & 4.260 & 4.260\tabularnewline
virginica & 6.588 & 2.974 & 5.552 & 5.552\tabularnewline
\bottomrule
\end{longtable}

Here is a sample \texttt{DT::datatable} output\footnote{cross
  referencing html tables isn't
  \href{https://github.com/lierdakil/pandoc-crossref/issues/122}{possible
  with pandoc-crossref}}.

\textbf{\texttt{DT::datatable}} only supported when rendering to html

Here we display a crosstab displayed in several different ways with a
``pills'' interface. To do this, just pass your \texttt{table()} result
to the \texttt{pilltabs()} function.

\textbf{\texttt{rmdformats::pilltabs()}} only supported when rendering
to html

\hypertarget{mathjax}{%
\subsection{Mathjax}\label{mathjax}}

An incredibly complex equation :

\begin{equation}x = \frac{-b \pm \sqrt{b^2-4ac}}{2a}\label{eq:quadratic_formula}\end{equation}

\hypertarget{figures}{%
\subsection{Figures}\label{figures}}

Here is an histogram.

\includegraphics{readthedown_files/figure-latex/iris_hist-1.pdf}

And a wonderful scatterplot, with a caption.

\begin{figure}
\hypertarget{fig:iris_scatter1}{%
\centering
\includegraphics{readthedown_files/figure-latex/iris_scatter1-1.pdf}
\caption{This is a caption}\label{fig:iris_scatter1}
}
\end{figure}

\hypertarget{sec:lb-support}{%
\section{Lightbox Plugin}\label{sec:lb-support}}

Using the \href{http://dimsemenov.com/plugins/magnific-popup/}{Magnific
popup} lightbox plugin for plots and images. To enable the plugin set
\texttt{lightbox:\ TRUE} and for gallery support \texttt{gallery:\ TRUE}
in the yaml header. To suppress the lightbox for a plot or image add the
class \texttt{"no-image-lb"}.

\hypertarget{caption}{%
\subsection{Caption}\label{caption}}

Extracts the html from the figure caption. You can use any inline
markdown syntax:

\begin{figure}
\hypertarget{fig:lb-caption}{%
\centering
\includegraphics{readthedown_files/figure-latex/lb-caption-1.pdf}
\caption[For \sout{more} \emph{Infos} \textbf{about} Violin plots see
\href{https://ggplot2.tidyverse.org/reference/geom_violin.html}{ggplot2
reference}. Inline markdown: Superscripts 2\textsuperscript{nd},
Subcripts H\textsubscript{2}O, inline TeX math \(2 + 2\), inline Code
\texttt{echo\ \textquotesingle{}hello\textquotesingle{}} and even
footnotes]{For \sout{more} \emph{Infos} \textbf{about} Violin plots see
\href{https://ggplot2.tidyverse.org/reference/geom_violin.html}{ggplot2
reference}. Inline markdown: Superscripts 2\textsuperscript{nd},
Subcripts H\textsubscript{2}O, inline TeX math \(2 + 2\), inline Code
\texttt{echo\ \textquotesingle{}hello\textquotesingle{}} and even
footnotes\footnotemark{}}\label{fig:lb-caption}
}
\end{figure}
\footnotetext{Footnote in figure caption}

\hypertarget{suppress-lightbox}{%
\subsection{Suppress lightbox}\label{suppress-lightbox}}

Adding the class \texttt{no-image-lb} to the image (with the pandoc
extension:
\href{https://pandoc.org/MANUAL.html\#other-extensions}{link\_attributes})
or plot (with
\texttt{out.extra=\textquotesingle{}class="no-image-lb"\textquotesingle{}})

\begin{figure}
\hypertarget{fig:ol-logo-no-lb}{%
\centering
\includegraphics{ostluft-logo.png}
\caption{Ostluft Logo without lightbox}\label{fig:ol-logo-no-lb}
}
\end{figure}

\begin{figure}
\hypertarget{fig:suppress-lb}{%
\centering
\includegraphics{readthedown_files/figure-latex/suppress-lb-1.pdf}
\caption{Number of cars in each class without
lightbox}\label{fig:suppress-lb}
}
\end{figure}

\end{document}